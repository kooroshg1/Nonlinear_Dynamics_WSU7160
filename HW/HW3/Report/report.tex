\documentclass[14pt, a4paper]{extreport}
\usepackage[margin=1.5cm]{geometry}
\usepackage{amsmath}
\usepackage{graphicx}
\usepackage{caption}
\usepackage{subcaption}
\usepackage{float}
\usepackage{wasysym}
\usepackage[procnames]{listings}
\usepackage{color}

\definecolor{keywords}{RGB}{255,0,90}
\definecolor{comments}{RGB}{0,0,113}
\definecolor{red}{RGB}{160,0,0}
\definecolor{green}{RGB}{0,150,0}
 
\lstset{language=Python, 
        basicstyle=\ttfamily\small, 
        keywordstyle=\color{keywords},
        commentstyle=\color{comments},
        stringstyle=\color{red},
        showstringspaces=false,
        identifierstyle=\color{green},
        procnamekeys={def,class}}
% ===============================================================
\begin{document}
\textbf{Problem definition:} Determine the decay envelope using the method of multiple scales.

%
\begin{equation}\label{eq:GE}
	\ddot{x} + \omega_0^2 x = \epsilon \left( \dot{x} - \dot{x}^3 \right)
\end{equation}
%

We define our differential operators as follow:

%
\begin{subequations}\label{eq:operators}
\begin{align}
	\frac{d}{dt^2} &= D_0^2 + 2 \epsilon D_0 D_1 + \epsilon^2 					\left(D_1^2 + 2 D_0 D_2\right)
	\\
	\frac{d}{dt} &= D_0 + \epsilon D_1
\end{align}
\end{subequations}
%

We approximate $x$ as follows:

%
\begin{equation}\label{eq:xApprox}
	x(t) = x_0 (T_0, T_1, T_2) + 
	       \epsilon x_1 (T_0, T_1, T_2)
\end{equation}
%

Substituting Equations \eqref{eq:operators} and \eqref{eq:xApprox} into \eqref{eq:GE} and grouping terms with respect to $\epsilon$ we get the following system of equations:

%
\begin{subequations}\label{eq:EOMeps}
\begin{align}
	\epsilon^0 : \quad D_{0}^{2} x_{0} + \omega_{0}^{2} x_{0} &= 0
	\label{eq:EOMeps0}
	\\
	\epsilon^1 : \quad D_{0}^{2} x_{1} + \omega_{0}^{2} x_{1} &= -
				 \left(
				 D_0^3 x_0^3 + 
				 2 D_{0} D_{1} x_{0} - 
				 D_0 x_0
				 \right)
	\label{eq:EOMeps1}
\end{align}
\end{subequations}
%

We assume the solution of Equation \eqref{eq:EOMeps0} as follows:

%
\begin{equation}\label{eq:x0}
	x_0 = A(T_1) \exp \left( i\omega_0 T_0 \right) + cc
\end{equation}
%

Substituting this equation into Equation \eqref{eq:EOMeps1} we get:

%
\begin{equation}
	D_{0}^{2} x_{1} + \omega_{0}^{2} x_{1} =
	\left[
	A(T_1) - \frac{dA(T_1)}{dT_1}
	\right]
	2i \exp \left( iT_0 \right)
\end{equation}
%

To remove the secular terms, the coefficient of the exponential needs to be equal to zero. By assuming $A(T_1)$ in the form of $a(T_1) \exp (ib(T_1))$ we get:

%
\begin{subequations}
\begin{align*}
	a &= \exp\left( T_1 \right) \\
	b' &= 0 \Rightarrow b = \mathcal{C}
\end{align*}
\end{subequations}
%

The particular solution of Equation \eqref{eq:EOMeps1} can be written as:

%
\begin{equation}\label{eq:x1}
	x_1 = A(T_1) \exp \left( i\omega_0 T_0 \right)
\end{equation}
%

Finally by substituting Equations \eqref{eq:x0} and \eqref{eq:x1} into \eqref{eq:xApprox}, the solution to \eqref{eq:GE} can be written as:

%
\begin{equation}
	x(t) = e^{T_1 + i\mathcal{C}} \exp \left( i\omega_0 T_0 \right) + 
	       \epsilon
	       \left[
	       e^{T_1 + i\mathcal{C}} \exp \left( i\omega_0 T_0 \right)
	       \right]
\end{equation}
% 

The decay envelop is:

%
\begin{equation}
	\left( 1 + \epsilon \right) e^{T_1 + i\mathcal{C}}
\end{equation}
%

This means that the amplitude of the vibrations increases. This is true due to negative damping coefficient in Equation \eqref{eq:GE}.
\end{document}